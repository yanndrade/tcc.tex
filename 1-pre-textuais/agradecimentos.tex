Primeiramente, agradeço a Deus, por ter me guiado, concedido saúde e forças para concluir esta etapa, e por ter estado comigo durante toda a minha vida.

À minha família, meu alicerce. Ao meu pai, por ser minha fonte de inspiração constante. À minha mãe, por todo o carinho e cuidado que sempre me dedicaram. Aos meus irmãos, que sempre me apoiaram e torceram pelo meu sucesso. Um agradecimento especial à Dorinha, figura extremamente importante na minha criação, e à memória de minha avó, que foi essencial na minha formação e que, com certeza, estaria feliz em ver onde seu neto chegou.

À minha noiva Maria Eduarda, minha maior fonte de inspiração. Obrigado por ter me suportado, me apoiado e caminhado ao meu lado em todas as etapas importantes da minha vida até agora, inclusive superando juntos mais este desafio.

Ao meu orientador, Prof. Dr. Dalton Honório de Araújo, e ao meu coorientador, Prof. Me. Paulo Honório Filho, por todo o suporte, apoio e incentivo fundamentais para o desenvolvimento deste trabalho.

Aos membros da banca examinadora, pela disponibilidade e pelas contribuições valiosas para a melhoria deste trabalho.

Aos amigos que a universidade me presenteou, pela amizade, pelas risadas, e pelas horas de estudo e dificuldades compartilhadas. Em especial a Gabriel, Rufino, Lívia e Kevyn, pela companhia leal desde o primeiro semestre do curso.

Aos amigos que fiz durante o intercâmbio, pela parceria inestimável. Sei que levarei vocês para toda a vida: Diego, Henrique, Tiago, David, Brasilino, Zé, Lorenzzon, Cilla, Kevin e Reis.

Aos projetos de extensão que foram fundamentais na minha formação pessoal e profissional: o PET Engenharia Elétrica, o Ramo Estudantil IEEE e o SIGHT UFC.

À Universidade Federal do Ceará (UFC) e ao seu corpo docente, onde pude adquirir os conhecimentos que tenho hoje e que me proporcionaram a base para ser quem sou.

À Coordenação de Aperfeiçoamento de Pessoal de Nível Superior (CAPES), pela concessão da bolsa BRAFITEC, que viabilizou a realização do meu intercâmbio na França, permitindo a realização de um sonho e uma experiência cultural e acadêmica imensurável.
