Este trabalho apresenta o desenvolvimento e a validação de um agente baseado em Modelos de Linguagem de Grande Escala (LLMs) e Recuperação Aumentada por Busca (RAG) para auxiliar no dimensionamento inicial de projetos elétricos residenciais de baixa tensão. A metodologia proposta integra a interação em linguagem natural e a interpretação de plantas baixas com rotinas determinísticas de cálculo, assegurando a conformidade com a norma ABNT NBR 5410 e diretrizes de concessionárias locais. O sistema foi estruturado para realizar o levantamento de cargas, a divisão de circuitos terminais, o dimensionamento de condutores e dispositivos de proteção, e a definição do padrão de fornecimento, gerando automaticamente um memorial de cálculo auditável. A validação foi conduzida por meio de estudos de caso baseados em exercícios didáticos de engenharia, demonstrando que o agente é capaz de produzir resultados consistentes, normativamente conformes e eletricamente coerentes, oferecendo uma ferramenta eficaz para o aumento da produtividade na etapa preliminar de projetos elétricos.

% Separe as palavras-chave por ponto
\palavraschave{Inteligência Artificial. Projeto Elétrico Residencial. NBR 5410. Modelos de Linguagem. RAG.}