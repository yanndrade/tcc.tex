\chapter{Introdução}
\label{chap:introducao}

O projeto de instalações elétricas residenciais de baixa tensão é uma atividade técnica que combina requisitos de segurança, critérios normativos e decisões de engenharia com impacto direto na confiabilidade e na manutenção da instalação. Mesmo quando o escopo é considerado ``inicial'' (por exemplo, a previsão de cargas mínimas, a organização de circuitos terminais e o dimensionamento de condutores e dispositivos de proteção), o processo envolve uma sequência de passos interdependentes, nos quais pequenas inconsistências podem gerar retrabalho ou decisões incoerentes ao longo do documento final.

Um produto central desse processo é o \textit{memorial de cálculo}, documento que registra as premissas adotadas, os critérios utilizados e os resultados obtidos em cada etapa de dimensionamento. Além de apoiar a comunicação técnica entre projetista, execução e fiscalização, o memorial é útil por tornar explícita a lógica das decisões: por que determinado circuito foi criado, por que uma seção foi escolhida, ou por que o tipo de fornecimento se enquadra em um padrão específico. Na prática, entretanto, a elaboração do memorial frequentemente se apoia em procedimentos manuais, consulta pontual a normas e uso de planilhas e modelos pré-formatados, o que aumenta o dispêndio de tempo na etapa preliminar e favorece inconsistências de rastreabilidade, como ocorre quando um valor é alterado e a justificativa textual não é atualizada.

No Brasil, a ABNT NBR 5410 estabelece requisitos mínimos e critérios de segurança para instalações elétricas de baixa tensão \cite{NBR5410:2004}, incluindo regras objetivas para previsão de cargas e pontos de utilização, bem como diretrizes associadas à organização de circuitos. Adicionalmente, diretrizes de concessionárias influenciam decisões como o tipo de fornecimento, limites de potência instalada e exigências de entrada de energia. Assim, mesmo em projetos residenciais, o dimensionamento inicial precisa equilibrar normatividade, consistência elétrica e clareza documental, sob risco de gerar um memorial pouco auditável ou tecnicamente frágil.

Em paralelo, os últimos anos trouxeram um avanço significativo em Modelos de Linguagem de Grande Escala (\textit{Large Language Models}, ou LLMs) e em técnicas de recuperação de informação, ampliando as possibilidades de sistemas capazes de interagir em linguagem natural e apoiar atividades técnicas \cite{Nguyen2025}. Esses modelos têm se mostrado eficazes para interpretar instruções, sintetizar textos e auxiliar a navegação em grandes volumes de informação, o que é promissor em atividades de engenharia com alta carga documental.

Apesar desse potencial, o uso direto de LLMs como ``motor de cálculo'' em problemas normativos apresenta limitações importantes: (i) a geração probabilística pode produzir respostas convincentes, porém incorretas (alucinações); (ii) o modelo não garante reprodutibilidade de resultados; (iii) há limitações de janela de contexto para inserir integralmente normas e diretrizes; e (iv) operações matemáticas e validações formais podem falhar quando tratadas apenas como geração de texto \cite{Ji2023}. Em engenharia, essas limitações são especialmente críticas porque o resultado precisa ser justificável, auditável e tecnicamente consistente.

Diante desse cenário, emerge uma lacuna: como aproveitar a capacidade de interação e interpretação dos LLMs, que é útil para levantar requisitos, organizar informações e redigir documentação, sem abrir mão do determinismo exigido pelos cálculos e pelas verificações normativas? Esta questão é particularmente relevante na etapa preliminar de projetos residenciais, em que há muitas regras ``mecânicas'' (previsões mínimas, critérios de divisão, seleção inicial de condutores e proteções) que podem ser formalizadas e executadas de forma determinística, ao mesmo tempo em que o usuário ou discente tende a fornecer dados em linguagem natural ou de forma incompleta.

A hipótese que orienta este trabalho é que uma arquitetura composta, na qual o LLM atua como camada de interpretação e orquestração, enquanto o cálculo e as checagens são executados por rotinas determinísticas, pode gerar memoriais de cálculo iniciais mais consistentes, rastreáveis e úteis como ponto de partida ao projetista. Para reduzir o risco de decisões ``desancoradas'', utiliza-se ainda a Recuperação Aumentada por Busca (RAG), permitindo que trechos normativos relevantes sejam recuperados e utilizados como base textual para justificativas e explicações \cite{Lewis2020}.

Assim, este trabalho investiga e avalia uma metodologia, implementada em um protótipo funcional, capaz de: (i) adquirir requisitos por conversação em linguagem natural e/ou interpretação de planta baixa; (ii) estruturar os dados do imóvel em um modelo consistente; (iii) executar um \textit{pipeline} determinístico para previsão de cargas, divisão em circuitos e dimensionamentos iniciais; e (iv) gerar automaticamente um memorial de cálculo com rastreabilidade das premissas e resultados. Essa abordagem busca equilibrar a flexibilidade da interação em linguagem natural com a segurança do cálculo normativo, servindo como ferramenta de apoio à etapa preliminar de projeto e como recurso didático para comparação com soluções de referência.

\section{Motivação}
A motivação central deste trabalho é demonstrar que é possível empregar técnicas modernas de inteligência artificial para apoiar a engenharia elétrica de forma responsável, preservando o caráter normativo do domínio. Em particular, busca-se contribuir para o aumento de produtividade na etapa inicial de projeto, reduzindo o tempo dedicado a tarefas repetitivas e fornecendo rapidamente um memorial de cálculo com rastreabilidade das premissas e resultados. 

Do ponto de vista acadêmico, a proposta também é relevante por aproximar o discente de um fluxo de projeto estruturado e por permitir comparações objetivas com soluções de referência, evidenciando invariantes normativas e variações legítimas por critério de projeto.

\section{Objetivos}
Este trabalho orienta-se pelos seguintes objetivos, divididos em um objetivo geral, que sintetiza o propósito principal da pesquisa, e objetivos específicos, que detalham as metas intermediárias necessárias para sua consecução.

\subsection{Objetivo geral}
Desenvolver e avaliar uma metodologia, implementada em um protótipo funcional, para um agente baseado em LLM com Recuperação Aumentada (RAG) capaz de realizar o dimensionamento inicial de instalações elétricas residenciais de baixa tensão e gerar automaticamente um memorial de cálculo, em conformidade com critérios aplicáveis da ABNT NBR 5410 e diretrizes locais consideradas.

\subsection{Objetivos específicos}
\begin{itemize}
    \item Estruturar a aquisição e consolidação de requisitos do projeto por meio de conversação em linguagem natural e/ou interpretação de planta baixa em formato de imagem.
    \item Construir um modelo estruturado do imóvel para armazenar ambientes, dimensões e cargas previstas, mantendo rastreabilidade entre entradas, resultados intermediários e saídas.
    \item Implementar um \textit{pipeline} determinístico para previsão de cargas mínimas, divisão em circuitos terminais, dimensionamento de condutores e dispositivos de proteção, e determinação do tipo de fornecimento.
    \item Realizar verificações de conformidade nas etapas do \textit{pipeline}, prevendo realimentação quando inconsistências forem detectadas.
    \item Validar o método por meio de estudos de caso didáticos, comparando resultados com referências acadêmicas (gabaritos e soluções manuais).
\end{itemize}

\section{Metodologia}
A pesquisa adota uma abordagem aplicada, com implementação de um protótipo funcional e avaliação por estudos de caso. O método integra: (i) aquisição de informações por texto e/ou imagem; (ii) consolidação das entradas em um modelo estruturado do imóvel; (iii) execução de rotinas determinísticas baseadas em regras normativas para cálculo de cargas, divisão de circuitos e dimensionamentos; e (iv) comparação dos resultados com materiais de referência do contexto didático. O detalhamento completo do método é apresentado no Capítulo \ref{chap:metodologia} e os resultados da validação no Capítulo \ref{chap:resultados}.

\section{Delimitação do escopo}
O escopo deste trabalho concentra-se no dimensionamento inicial de instalações residenciais de baixa tensão, contemplando cargas mínimas de iluminação e tomadas de uso geral, divisão em circuitos terminais, dimensionamento de condutores e dispositivos de proteção por critérios de capacidade de condução de corrente e coordenação entre proteção e condutor, e determinação do tipo de fornecimento conforme diretrizes locais consideradas. Aspectos como queda de tensão, curto-circuito, DR, DPS, aterramento e diagramas unifilares são tratados como extensões futuras.

\section{Estrutura do trabalho}
Este trabalho está organizado em cinco capítulos. O Capítulo 1 apresenta a contextualização, motivação, objetivos, metodologia e estrutura do texto. O Capítulo 2 discute a fundamentação teórica necessária, incluindo normas e conceitos relacionados a instalações residenciais e aos princípios de agentes baseados em LLMs e RAG. O Capítulo 3 descreve a metodologia proposta. O Capítulo 4 apresenta os resultados e a validação por estudos de caso. Por fim, o Capítulo 5 apresenta as conclusões, limitações e trabalhos futuros.

