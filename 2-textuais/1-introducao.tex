\chapter{Introdução}
\label{chap:introducao}

Nos últimos anos, modelos de linguagem de grande porte (\textit{Large Language Models} -- LLMs) e técnicas de recuperação de informação têm ampliado as possibilidades de sistemas capazes de interagir em linguagem natural e apoiar atividades técnicas. Em diferentes áreas da engenharia, observa-se um movimento de adoção de soluções baseadas em inteligência artificial para acelerar etapas repetitivas, reduzir retrabalho e melhorar a rastreabilidade de decisões. 

No contexto de instalações elétricas residenciais de baixa tensão, uma parte relevante do trabalho de projeto envolve a consolidação de requisitos do imóvel, o levantamento de cargas mínimas por ambiente, a organização em circuitos terminais e o dimensionamento inicial de condutores e dispositivos de proteção, sempre sob restrições normativas. No Brasil, a ABNT NBR 5410 estabelece critérios mínimos e requisitos de segurança para instalações elétricas de baixa tensão, incluindo regras objetivas para previsão de cargas e pontos de utilização, além de diretrizes associadas à organização dos circuitos. Adicionalmente, diretrizes da concessionária podem influenciar decisões como o tipo de fornecimento aplicável e critérios de entrada de energia.

Apesar do potencial de LLMs em tarefas de linguagem, o uso direto desses modelos para dimensionamento elétrico apresenta desafios, especialmente quanto à necessidade de conformidade normativa, reprodutibilidade e justificativa técnica. Dessa forma, este trabalho investiga uma abordagem que combina interação em linguagem natural e recuperação de conhecimento normativo (RAG) com rotinas determinísticas de cálculo e verificação, visando produzir resultados consistentes, auditáveis e úteis como ponto de partida para o projetista.

\section{Motivação}
A motivação central deste trabalho é demonstrar que é possível empregar técnicas modernas de inteligência artificial para apoiar a engenharia elétrica de forma responsável, preservando o caráter normativo do domínio. Em particular, busca-se contribuir para o aumento de produtividade na etapa inicial de projeto, reduzindo o tempo dedicado a tarefas repetitivas e fornecendo rapidamente um memorial de cálculo com rastreabilidade das premissas e resultados. 

Do ponto de vista acadêmico, a proposta também é relevante por aproximar o discente de um fluxo de projeto estruturado e por permitir comparações objetivas com soluções de referência, evidenciando invariantes normativas e variações legítimas por critério de projeto.

\section{Objetivos}

\subsection{Objetivo geral}
Desenvolver e avaliar uma metodologia, implementada em um protótipo funcional, para um agente baseado em LLM com Recuperação Aumentada (RAG) capaz de realizar o dimensionamento inicial de instalações elétricas residenciais de baixa tensão e gerar automaticamente um memorial de cálculo, em conformidade com critérios aplicáveis da ABNT NBR 5410 e diretrizes locais consideradas.

\subsection{Objetivos específicos}
\begin{itemize}
    \item Estruturar a aquisição e consolidação de requisitos do projeto por meio de conversação em linguagem natural e/ou interpretação de planta baixa em formato de imagem.
    \item Construir um modelo estruturado do imóvel para armazenar ambientes, dimensões e cargas previstas, mantendo rastreabilidade entre entradas, resultados intermediários e saídas.
    \item Implementar um pipeline determinístico para previsão de cargas mínimas, divisão em circuitos terminais, dimensionamento de condutores e dispositivos de proteção, e determinação do tipo de fornecimento.
    \item Realizar verificações de conformidade nas etapas do pipeline, prevendo realimentação quando inconsistências forem detectadas.
    \item Validar o método por meio de estudos de caso didáticos, comparando resultados com referências acadêmicas (gabaritos e soluções manuais).
\end{itemize}

\section{Metodologia}
A pesquisa adota uma abordagem aplicada, com implementação de um protótipo funcional e avaliação por estudos de caso. O método integra: (i) aquisição de informações por texto e/ou imagem; (ii) consolidação das entradas em um modelo estruturado do imóvel; (iii) execução de rotinas determinísticas baseadas em regras normativas para cálculo de cargas, divisão de circuitos e dimensionamentos; e (iv) comparação dos resultados com materiais de referência do contexto didático. O detalhamento completo do método é apresentado no Capítulo \ref{chap:metodologia} e os resultados da validação no Capítulo \ref{chap:resultados}.

\section{Delimitação do escopo}
O escopo deste trabalho concentra-se no dimensionamento inicial de instalações residenciais de baixa tensão, contemplando cargas mínimas de iluminação e tomadas de uso geral, divisão em circuitos terminais, dimensionamento de condutores e dispositivos de proteção por critérios de capacidade de condução de corrente e coordenação entre proteção e condutor, e determinação do tipo de fornecimento conforme diretrizes locais consideradas. Aspectos como queda de tensão, curto-circuito, DR, DPS, aterramento e diagramas unifilares são tratados como extensões futuras.

\section{Estrutura do trabalho}
Este trabalho está organizado em cinco capítulos. O Capítulo 1 apresenta a contextualização, motivação, objetivos, metodologia e estrutura do texto. O Capítulo 2 discute a fundamentação teórica necessária, incluindo normas e conceitos relacionados a instalações residenciais e aos princípios de agentes baseados em LLMs e RAG. O Capítulo 3 descreve a metodologia proposta. O Capítulo 4 apresenta os resultados e a validação por estudos de caso. Por fim, o Capítulo 5 apresenta as conclusões, limitações e trabalhos futuros.

