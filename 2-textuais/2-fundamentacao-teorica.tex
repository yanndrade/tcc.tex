\chapter{Fundamentação Teórica}
\label{chap:fundamentacao_teorica}

Este capítulo reúne os principais conceitos necessários para compreender a proposta e os
resultados deste trabalho. Inicialmente, são discutidos fundamentos relacionados a instalações
elétricas residenciais de baixa tensão, com ênfase no papel do \textit{memorial de cálculo} como
documento técnico de consolidação do raciocínio de projeto. Em seguida, são apresentados os
fundamentos normativos empregados no levantamento de cargas mínimas, na divisão em circuitos
terminais e nos critérios de dimensionamento de condutores e dispositivos de proteção, tomando
como base a ABNT NBR 5410 \cite{NBR5410} e as diretrizes da concessionária consideradas no
estudo \cite{ET124_ENEL}.

Além da fundamentação normativa, este capítulo também introduz os conceitos computacionais
que viabilizam a automatização parcial do processo de elaboração do memorial de cálculo,
incluindo noções de Modelos de Linguagem de Grande Escala (LLMs), agentes com uso de
ferramentas e Recuperação Aumentada por Busca (RAG). Destaca-se, ainda, a separação entre:
(i) componentes determinísticos, responsáveis pelos cálculos e pela aplicação consistente das
regras; e (ii) componentes linguísticos, responsáveis por organizar, explicar e apresentar os
resultados de forma coerente e rastreável ao usuário. Por fim, são contextualizados os fundamentos
associados à interpretação de informações a partir de plantas em imagem e suas limitações
práticas, uma vez que a qualidade e precisão da entrada impactam diretamente as estimativas
utilizadas nas etapas de cálculo.

\section{Instalações elétricas residenciais e memorial de cálculo}
\label{sec:ft_instalacoes_memorial}

\subsection{Objetivo do projeto elétrico em Baixa Tensão (BT)}

Uma instalação elétrica residencial de baixa tensão pode ser entendida como o conjunto de
circuitos, condutores, dispositivos de proteção, pontos de utilização e quadros de distribuição
destinados a atender, com segurança e funcionalidade, as demandas de energia elétrica de uma
unidade habitacional. Em termos de projeto, busca-se garantir o suprimento adequado das
cargas previstas, assegurando simultaneamente requisitos fundamentais como proteção contra
sobrecorrentes, limitação de efeitos térmicos, organização e manutenção, conforme práticas
normativas e de engenharia adotadas no contexto nacional.

No processo tradicional de elaboração de um projeto elétrico residencial, parte-se de um
levantamento de cargas, que pode ser baseado em prescrições mínimas normativas e/ou em
equipamentos efetivamente previstos, para então definir a potência instalada e proceder à
divisão da instalação em circuitos terminais. Essa divisão impacta diretamente a organização do
quadro de distribuição, a seleção dos condutores, a escolha dos dispositivos de proteção e, quando
aplicável, a distribuição de cargas entre fases.

\subsection{O que é memorial e por que ele é útil}

O \textit{memorial de cálculo} representa um documento técnico que consolida, de forma estruturada,
os principais elementos do raciocínio de projeto. Em geral, ele registra:
(i) hipóteses e critérios adotados; (ii) levantamento e composição de cargas; (iii) divisão em
circuitos; (iv) dimensionamento de condutores; (v) dimensionamento e seleção de dispositivos
de proteção; e (vi) sínteses e tabelas finais que facilitem a verificação e o uso posterior do documento.
Neste trabalho, o memorial de cálculo é o principal artefato de saída do sistema proposto e,
na versão atual, é gerado automaticamente em formato de documento \textbf{Microsoft Word}.

A utilidade do memorial de cálculo é múltipla. Do ponto de vista técnico, ele fornece
\textit{rastreabilidade}: permite que um terceiro compreenda quais regras, premissas e verificações
levaram aos valores e escolhas apresentados. Do ponto de vista de validação, o memorial facilita
a checagem de coerência entre carga prevista, circuitos definidos, seções de condutores e proteções
especificadas, reduzindo ambiguidades e erros de comunicação. Em cenários acadêmicos e
profissionais, esse tipo de registro também tem valor por tornar o projeto \textit{auditável}, isto é,
verificável a partir de critérios explicitados.

\section{Norma ABNT NBR 5410: previsão de cargas mínimas}
\label{sec:ft_nbr5410_cargas_minimas}

Uma etapa essencial do projeto elétrico residencial é a \textbf{previsão de cargas mínimas}
associadas a \textit{iluminação} e \textit{tomadas de uso geral}. Na ausência de um levantamento
completo baseado em equipamentos efetivamente definidos, a ABNT NBR 5410 estabelece
\textbf{critérios prescritivos mínimos} que permitem estimar a carga instalada de forma
padronizada e consistente \cite{NBR5410}.

De modo geral, a norma utiliza valores em \textbf{VA} para essas previsões mínimas, pois se trata
de uma estimativa de potência aparente associada a pontos de utilização típicos.

\subsection{Iluminação (regras por área)}
\label{subsec:ft_nbr5410_iluminacao}

Para circuitos de iluminação, a ABNT NBR 5410 define uma carga mínima por cômodo ou
dependência em função de sua área \cite{NBR5410}. Em termos práticos:

\begin{itemize}
  \item \textbf{Ambientes com área $A \leq 6\ \text{m}^2$:} deve ser prevista \textbf{carga mínima de 100 VA}.
  \item \textbf{Ambientes com área $A > 6\ \text{m}^2$:} deve ser prevista \textbf{carga mínima de 100 VA}
  para os primeiros $6\ \text{m}^2$, acrescida de \textbf{60 VA} para cada \textbf{acréscimo de $4\ \text{m}^2$}
  inteiros além dos $6\ \text{m}^2$.
\end{itemize}

Uma forma compacta de representar a regra, para $A>6\ \text{m}^2$, é:
$$ P_{\text{ilum}} = 100 + 60 \cdot \left\lfloor \frac{A-6}{4} \right\rfloor \quad [\text{VA}] $$
onde $\lfloor \cdot \rfloor$ denota a parte inteira (isto é, considera-se apenas incrementos completos
de $4\ \text{m}^2$).

Além da potência mínima, a norma também estabelece a necessidade de previsão de pelo menos
\textbf{um ponto de luz fixo no teto} por cômodo ou dependência.

\subsection{Tomadas de uso geral (TUG)}
\label{subsec:ft_nbr5410_tug}

Para tomadas de uso geral, a ABNT NBR 5410 prescreve quantidades mínimas de pontos e
potências mínimas a serem atribuídas, variando conforme o tipo de ambiente \cite{NBR5410}.

\subsubsection{Banheiros}
Deve ser previsto \textbf{no mínimo um ponto de tomada} próximo ao lavatório. Para efeito de previsão, adota-se \textbf{600 VA por tomada}.

\subsubsection{Cozinhas, copas, áreas de serviço e locais semelhantes}
Deve ser previsto \textbf{no mínimo um ponto de tomada para cada $3{,}5$ m, ou fração, de perímetro}.
Para atribuição de potência:
\begin{itemize}
  \item \textbf{600 VA} por ponto, \textbf{até três pontos};
  \item \textbf{100 VA} por ponto, para os \textbf{excedentes}.
\end{itemize}

\subsubsection{Salas e dormitórios}
Deve ser previsto \textbf{pelo menos um ponto de tomada para cada $5$ m, ou fração, de perímetro}.
Para previsão, adota-se \textbf{100 VA por tomada}.

\subsubsection{Demais cômodos e dependências}
Para outros ambientes:
\begin{itemize}
  \item Área $\le 6\ \text{m}^2$: \textbf{ao menos um ponto};
  \item Área $> 6\ \text{m}^2$: \textbf{um ponto a cada $5$ m, ou fração, de perímetro}.
\end{itemize}
Em ambos os casos, atribui-se \textbf{100 VA por tomada}.

\subsection{TUE e circuitos dedicados}
\label{subsec:ft_nbr5410_tue}

Além das TUG, a ABNT NBR 5410 estabelece que pontos destinados a alimentar, de modo
exclusivo ou virtualmente dedicado, equipamentos com \textbf{corrente nominal superior a 10 A}
devem constituir \textbf{circuito independente} \cite{NBR5410}. Isso caracteriza as \textbf{tomadas de uso específico (TUE)}.

Essa prescrição reduz interferências e evita sobrecarga de circuitos de uso geral. Além disso, mesmo para equipamentos com corrente inferior a 10 A, pode ser recomendável (a critério do projetista) a adoção de circuitos exclusivos quando houver características específicas (motores, cargas sensíveis, uso contínuo), demonstrando que existe um espaço de \textbf{critério de projetista} para decidir quando a segregação é desejável.

\section{Divisão em circuitos terminais (NBR 5410 + boas práticas)}
\label{sec:ft_divisao_circuitos}

Após a previsão das cargas, a instalação deve ser \textbf{dividida em circuitos terminais}.
A ABNT NBR 5410 estabelece essa divisão como requisito para segurança, manutenção e redução de interferências.

\subsection{Separação entre iluminação e TUG}
\label{subsec:ft_sep_ilum_tomadas}

Como diretriz geral e boa prática amplamente aceita, recomenda-se manter \textbf{circuitos de iluminação separados dos circuitos de tomadas}, evitando que falhas em cargas de tomadas provoquem desligamentos de iluminação.
Essa segregação facilita também a manutenção e o diagnóstico de falhas.

\subsection{Exclusividade para cozinha, área de serviço e TUE}
\label{subsec:ft_exclusividade_cozinha}

A norma estabelece situações em que circuitos exclusivos são necessários:
\begin{itemize}
  \item \textbf{TUE (Corrente > 10 A):} Devem constituir circuitos independentes.
  \item \textbf{Cozinhas, áreas de serviço e afins:} Os pontos de tomada desses ambientes devem ser atendidos por \textbf{circuitos exclusivamente destinados}
a eles \cite{NBR5410}.
\end{itemize}

\subsection{Limites típicos por circuito (boas práticas)}
\label{subsec:ft_limites_potencia}

Embora a norma defina requisitos mínimos, é prática comum adotar limites de potência para evitar correntes elevadas em circuitos de uso geral e facilitar a proteção. No contexto deste trabalho, consideram-se como \textbf{boas práticas}:

\begin{itemize}
  \item \textbf{Iluminação:} Limitar a potência para evitar desligamento de grandes áreas em caso de falha. Em 220 V, valores da ordem de 2500 VA são usuais.
  \item \textbf{TUG:} Limitar a potência total para manter correntes compatíveis com disjuntores de uso geral (ex.: 16 A ou 20 A). Em 220 V, limites da ordem de 4300 VA são referências comuns de projeto.
\end{itemize}

\subsection{Circuitos de reserva}
\label{subsec:ft_circuitos_reserva}

A ABNT NBR 5410 estabelece que os quadros de distribuição devem prever \textbf{capacidade de reserva} (espaço físico) para ampliações futuras \cite{NBR5410}.
Na prática de projeto, isso se materializa prevendo \textbf{circuitos reserva} no quadro, dimensionados e deixados vagos para uso futuro. A quantidade de reservas depende do número de circuitos efetivos (ex.: para quadros com até 6 circuitos, prever no mínimo 2 reservas).

\section{Dimensionamento de condutores e proteções (NBR 5410)}
\label{sec:ft_dimensionamento_condutores}

Uma vez definidos os circuitos, procede-se ao \textbf{dimensionamento dos condutores} e à \textbf{seleção dos dispositivos de proteção}.

\subsection{Corrente de projeto, ampacidade e seção mínima}
\label{subsec:ft_corrente_ampacidade}

\paragraph{Corrente de projeto ($I_B$):} Calculada a partir da potência do circuito e da tensão ($I_B = S/V$).\paragraph{Ampacidade ($I_Z$):} A capacidade de condução de corrente do condutor depende do método de instalação, material, isolação e agrupamento. A seção deve ser tal que $I_Z \ge I_B$ (considerando fatores de correção).\paragraph{Seção mínima:} A norma impõe seções mínimas por razões mecânicas, independentemente da corrente calculada:
\begin{itemize}
  \item \textbf{$1{,}5~\mathrm{mm}^2$} para iluminação;
  \item \textbf{$2{,}5~\mathrm{mm}^2$} para circuitos de força (tomadas).
\end{itemize}

\subsection{Critério de coordenação: $I_B \le I_N \le I_Z$}
\label{subsec:ft_criterio_disjuntor}

Para proteção contra sobrecarga, deve-se atender à relação fundamental de coordenação:
$$ I_B \le I_N \le I_Z $$
onde $I_B$ é a corrente de projeto, $I_N$ é a corrente nominal do disjuntor e $I_Z$ é a capacidade de condução do condutor nas condições instaladas. Isso garante que o disjuntor não dispare em operação normal ($I_B \le I_N$) e que proteja o cabo antes que este sobreaqueça ($I_N \le I_Z$).

\subsection{Condutores de Proteção (PE) e Neutro}
\label{subsec:ft_pe_neutro}

\paragraph{PE (Terra):} Prover caminho de baixa impedância para faltas. Sua seção é definida normativamente com base na seção do condutor de fase (geralmente igual até $16~\mathrm{mm}^2$).\paragraph{Neutro:} Condutor de retorno. Em circuitos monofásicos, sua seção é, em regra, igual à do condutor de fase.

\subsection{Hipóteses adotadas no escopo (220 V, B1, PVC)}
\label{subsec:ft_hipoteses}

Para viabilizar a automatização determinística neste trabalho, são adotadas hipóteses de projeto usuais para residências de padrão popular/médio na região do estudo:
\begin{itemize}
    \item Tensão de fase-neutro/fase-fase conforme concessionária (ex.: 220 V);
    \item Método de instalação \textbf{B1} (condutores em eletroduto embutido em alvenaria);
    \item Condutores de \textbf{Cobre} com isolação \textbf{PVC} ($70^{\circ}$C);
    \item Consideração típica de \textbf{3 condutores carregados} em agrupamentos iniciais (F+N ou F+F, agrupados), ajustável conforme o caso.
\end{itemize}

\section{Tipo de fornecimento e diretrizes da concessionária (ET-124 / ENEL Ceará)}
\label{sec:ft_tipo_fornecimento}

Além da NBR 5410, o projeto deve respeitar as normas da concessionária local (ET-124 da ENEL Ceará) \cite{ET124}.

\subsection{Enquadramento do fornecimento pela potência instalada}
\label{subsec:ft_enquadramento}

A ET-124 define se o fornecimento será \textbf{monofásico, bifásico ou trifásico} com base na \textbf{potência instalada total} da unidade consumidora.
A potência instalada é a soma das potências atribuídas a todas as cargas (iluminação, TUG e TUE). Faixas de potência (ex.: até 15 kW, até 25 kW) determinam o tipo de atendimento.

\subsection{Impactos do tipo de fornecimento no projeto}
\label{subsec:ft_impacto_fornecimento}

O tipo de fornecimento afeta:
\begin{enumerate}
    \item A arquitetura do quadro (monofásico vs. trifásico);
    \item O padrão de entrada e medição;
    \item A necessidade de \textbf{balanceamento de fases} na distribuição dos circuitos, visando equilibrar as cargas entre as fases disponíveis.
\end{enumerate}

\section{IA: Modelos de Linguagem (LLM)}
\label{sec:ft_llm}

\subsection{Conceito, capacidades e limitações}

Modelos de Linguagem de Grande Escala (LLMs) são sistemas treinados para prever e gerar texto de forma coerente. Eles são capazes de interpretar linguagem natural, resumir informações e estruturar dados. Contudo, operam de forma probabilística.

\subsection{Por que LLM sozinho não serve para cálculo normativo}

O uso isolado de LLMs para engenharia apresenta riscos:
\begin{itemize}
    \item \textbf{Alucinações:} Geração de informações incorretas com alta confiança;
    \item \textbf{Inconsistência:} Falta de repetibilidade nos cálculos;
    \item \textbf{Dificuldade matemática:} Erros em operações aritméticas complexas.
\end{itemize}
Para projetos elétricos, onde a segurança depende de valores exatos e conformidade normativa, essas limitações inviabilizam o uso do LLM como única "calculadora".

\section{Agentes e uso de ferramentas}
\label{sec:ft_agentes}

\subsection{Conceito de agente}

Um agente é um sistema que utiliza um LLM como cérebro para raciocinar, mas possui a capacidade de \textbf{agir} sobre o ambiente. Ele recebe um objetivo, planeja passos e executa ações.

\subsection{Ferramentas determinísticas para cálculos normativos}

Para mitigar as limitações dos LLMs, o agente utiliza \textbf{ferramentas determinísticas} (código convencional) para executar cálculos.
\begin{quote}
\textit{LLM (Interpretação/Texto)} + \textit{Ferramenta (Cálculo/Norma)} = \textit{Projeto Confiável}.
\end{quote}
Isso garante que, para uma dada entrada, o dimensionamento (seções, disjuntores) siga rigorosamente a NBR 5410.

\subsection{Validação e coerência}

O sistema valida se os resultados das ferramentas fazem sentido técnico (ex.: disjuntor não pode ser maior que a ampacidade do cabo) e se estão coerentes com a descrição do usuário, garantindo um projeto seguro e auditável.

\section{Recuperação Aumentada por Busca (RAG) aplicada a normas}
\label{sec:ft_rag}

\subsection{Justificativa para uso de RAG em normas}

Normas técnicas são extensas e detalhadas. O uso de RAG permite que o agente consulte trechos específicos da norma durante o processo, fundamentando suas decisões em texto real em vez de confiar apenas na memória de treinamento do modelo.

\subsection{Pipeline conceitual: indexação, recuperação e síntese}

\begin{enumerate}
    \item \textbf{Indexação:} Fragmentação das normas em trechos consultáveis;
    \item \textbf{Recuperação:} Busca híbrida (semântica + palavras-chave) por trechos relevantes ao contexto atual;
    \item \textbf{Síntese:} O agente usa a informação recuperada para tomar decisões ou explicar o memorial.
\end{enumerate}

\section{Síntese do capítulo}
\label{sec:ft_sintese}

Este capítulo estabeleceu a base para o desenvolvimento do sistema proposto. A amarração entre as \textbf{normas técnicas} (NBR 5410 e ET-124), que fornecem os critérios de engenharia, e a \textbf{arquitetura computacional} (Agentes, LLM e RAG), que provê a inteligência e flexibilidade, é o cerne da metodologia.
A estratégia de separar o determinismo (cálculo) da fluidez (texto) é fundamental para garantir que o memorial de cálculo gerado seja, ao mesmo tempo, tecnicamente rigoroso e acessível ao usuário.