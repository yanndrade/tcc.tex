\chapter{Fundamentação Teórica}
\label{chap:fundamentacao_teorica}

Este capítulo reúne os principais conceitos necessários para compreender a proposta e os
resultados deste trabalho. Inicialmente, são discutidos fundamentos relacionados a instalações
elétricas residenciais de baixa tensão, com ênfase no papel do \textit{memorial de cálculo} como
documento técnico de consolidação do raciocínio de projeto. Em seguida, são apresentados os
fundamentos normativos empregados no levantamento de cargas mínimas, na divisão em circuitos
terminais e nos critérios de dimensionamento de condutores e dispositivos de proteção, tomando
como base a ABNT NBR 5410 \cite{NBR5410:2004, NBR5410:2008} e as diretrizes da concessionária consideradas no
estudo \cite{ET124:ENEL}.

Além da fundamentação normativa, este capítulo também introduz os conceitos computacionais
que viabilizam a automatização parcial do processo de elaboração do memorial de cálculo,
incluindo noções de Modelos de Linguagem de Grande Escala (LLMs), agentes com uso de
ferramentas e Recuperação Aumentada por Busca (RAG). Destaca-se, ainda, a separação entre:
(i) componentes determinísticos, responsáveis pelos cálculos e pela aplicação consistente das
regras; e (ii) componentes linguísticos, responsáveis por organizar, explicar e apresentar os
resultados de forma coerente e rastreável ao usuário. Por fim, são contextualizados os fundamentos
associados à interpretação de informações a partir de plantas em imagem e suas limitações
práticas, uma vez que a qualidade e precisão da entrada impactam diretamente as estimativas
utilizadas nas etapas de cálculo.

\section{Instalações elétricas residenciais e memorial de cálculo}
\label{sec:ft_instalacoes_memorial}

\subsection{Objetivo do projeto elétrico em Baixa Tensão (BT)}

Uma instalação elétrica residencial de baixa tensão pode ser entendida como o conjunto de
circuitos, condutores, dispositivos de proteção, pontos de utilização e quadros de distribuição
destinados a atender, com segurança e funcionalidade, as demandas de energia elétrica de uma
unidade habitacional. Em termos de projeto, busca-se garantir o suprimento adequado das
cargas previstas, assegurando simultaneamente requisitos fundamentais como proteção contra
sobrecorrentes, limitação de efeitos térmicos, organização e manutenção, conforme práticas
normativas e de engenharia adotadas no contexto nacional \cite{Creder2016}.

No processo tradicional de elaboração de um projeto elétrico residencial, parte-se de um
levantamento de cargas, que pode ser baseado em prescrições mínimas normativas e/ou em
equipamentos efetivamente previstos, para então definir a potência instalada e proceder à
divisão da instalação em circuitos terminais. Essa divisão impacta diretamente a organização do
quadro de distribuição, a seleção dos condutores, a escolha dos dispositivos de proteção e, quando
aplicável, a distribuição de cargas entre fases.

\subsection{O que é memorial e por que ele é útil}

O \textit{memorial de cálculo} representa um documento técnico que consolida, de forma estruturada,
os principais elementos do raciocínio de projeto. Em geral, ele registra:
(i) hipóteses e critérios adotados; (ii) levantamento e composição de cargas; (iii) divisão em
circuitos; (iv) dimensionamento de condutores; (v) dimensionamento e seleção de dispositivos
de proteção; e (vi) sínteses e tabelas finais que facilitem a verificação e o uso posterior do documento.
Neste trabalho, o memorial de cálculo é o principal artefato de saída do sistema proposto e,
na versão atual, é gerado automaticamente em formato de documento Microsoft Word.

A utilidade do memorial de cálculo é múltipla. Do ponto de vista técnico, ele fornece
rastreabilidade: permite que um terceiro compreenda quais regras, premissas e verificações
levaram aos valores e escolhas apresentados. Do ponto de vista de validação, o memorial facilita
a checagem de coerência entre carga prevista, circuitos definidos, seções de condutores e proteções
especificadas, reduzindo ambiguidades e erros de comunicação \cite{Mamede2017}. Em cenários acadêmicos e
profissionais, esse tipo de registro também tem valor por tornar o projeto auditável, isto é,
verificável a partir de critérios explicitados.

\section{Norma ABNT NBR 5410: previsão de cargas mínimas}
\label{sec:ft_nbr5410_cargas_minimas}

Uma etapa essencial do projeto elétrico residencial é a previsão de cargas mínimas
associadas a \textit{iluminação} e \textit{tomadas de uso geral}. Na ausência de um levantamento
completo baseado em equipamentos efetivamente definidos, a ABNT NBR 5410 estabelece
critérios prescritivos mínimos que permitem estimar a carga instalada de forma
padronizada e consistente \cite{NBR5410:2004}.

De modo geral, a norma utiliza valores em Volt-Ampere (VA) para essas previsões mínimas, pois se trata
de uma estimativa de potência aparente associada a pontos de utilização típicos.

\subsection{Iluminação (regras por área)}
\label{subsec:ft_nbr5410_iluminacao}

Para circuitos de iluminação, a ABNT NBR 5410 define uma carga mínima por cômodo ou
dependência em função de sua área \cite{NBR5410:2004}. Em termos práticos:

\begin{itemize}
  \item Ambientes com área $A \leq 6\ \text{m}^2$: deve ser prevista carga mínima de 100 VA.
  \item Ambientes com área $A > 6\ \text{m}^2$: deve ser prevista carga mínima de 100 VA
  para os primeiros $6\ \text{m}^2$, acrescida de 60 VA para cada acréscimo de $4\ \text{m}^2$
  inteiros além dos $6\ \text{m}^2$.
\end{itemize}

Uma forma compacta de representar a regra, para $A>6\ \text{m}^2$, é:
$$ P_{\text{ilum}} = 100 + 60 \cdot \left\lfloor \frac{A-6}{4} \right\rfloor \quad [\text{VA}] $$
onde $\lfloor \cdot \rfloor$ denota a parte inteira (isto é, considera-se apenas incrementos completos
de $4\ \text{m}^2$).

Além da potência mínima, a norma também estabelece a necessidade de previsão de pelo menos
um ponto de luz fixo no teto por cômodo ou dependência.

\subsection{Tomadas de uso geral (TUG)}
\label{subsec:ft_nbr5410_tug}

Para tomadas de uso geral, a ABNT NBR 5410 prescreve quantidades mínimas de pontos e
potências mínimas a serem atribuídas, variando conforme o tipo de ambiente \cite{NBR5410:2004}.

\subsubsection{Banheiros}
Deve ser previsto no mínimo um ponto de tomada próximo ao lavatório. Para efeito de previsão, adota-se 600 VA por tomada.

\subsubsection{Cozinhas, copas, áreas de serviço e locais semelhantes}
Deve ser previsto no mínimo um ponto de tomada para cada $3{,}5$ m, ou fração, de perímetro.
Para atribuição de potência:
\begin{itemize}
  \item 600 VA por ponto, até três pontos;
  \item 100 VA por ponto, para os excedentes.
\end{itemize}

\subsubsection{Salas e dormitórios}
Deve ser previsto pelo menos um ponto de tomada para cada $5$ m, ou fração, de perímetro.
Para previsão, adota-se 100 VA por tomada.

\subsubsection{Demais cômodos e dependências}
Para outros ambientes:
\begin{itemize}
  \item Área $\le 6\ \text{m}^2$: ao menos um ponto;
  \item Área $> 6\ \text{m}^2$: um ponto a cada $5$ m, ou fração, de perímetro.
\end{itemize}
Em ambos os casos, atribui-se 100 VA por tomada.

\subsection{TUE e circuitos dedicados}
\label{subsec:ft_nbr5410_tue}

Além das TUG, a ABNT NBR 5410 estabelece que pontos destinados a alimentar, de modo
exclusivo ou virtualmente dedicado, equipamentos com corrente nominal superior a 10 A
devem constituir circuito independente \cite{NBR5410:2004}. Isso caracteriza as tomadas de uso específico (TUE).

Essa prescrição reduz interferências e evita sobrecarga de circuitos de uso geral. Além disso, mesmo para equipamentos com corrente inferior a 10 A, pode ser recomendável (a critério do projetista) a adoção de circuitos exclusivos quando houver características específicas (motores, cargas sensíveis, uso contínuo), demonstrando que existe um espaço de critério de projetista para decidir quando a segregação é desejável.

\section{Divisão em circuitos terminais (NBR 5410 + boas práticas)}
\label{sec:ft_divisao_circuitos}

Após a previsão das cargas, a instalação deve ser dividida em circuitos terminais.
A ABNT NBR 5410 estabelece essa divisão como requisito para segurança, manutenção e redução de interferências.

\subsection{Separação entre iluminação e TUG}
\label{subsec:ft_sep_ilum_tomadas}

Como diretriz geral e boa prática amplamente aceita, recomenda-se manter circuitos de iluminação separados dos circuitos de tomadas, evitando que falhas em cargas de tomadas provoquem desligamentos de iluminação \cite{Creder2016}.
Essa segregação facilita também a manutenção e o diagnóstico de falhas.

\subsection{Exclusividade para cozinha, área de serviço e TUE}
\label{subsec:ft_exclusividade_cozinha}

A norma estabelece situações em que circuitos exclusivos são necessários:
\begin{itemize}
  \item \textit{TUE (Corrente > 10 A):} Devem constituir circuitos independentes.
  \item \textit{Cozinhas, áreas de serviço e afins:} Os pontos de tomada desses ambientes devem ser atendidos por circuitos exclusivamente destinados a eles \cite{NBR5410:2004}.
\end{itemize}

\subsection{Limites típicos por circuito (boas práticas)}
\label{subsec:ft_limites_potencia}

Embora a norma defina requisitos mínimos, é prática comum adotar limites de potência para evitar correntes elevadas em circuitos de uso geral e facilitar a proteção. No contexto deste trabalho, consideram-se como boas práticas:

\begin{itemize}
  \item \textit{Iluminação:} Limitar a potência para evitar desligamento de grandes áreas em caso de falha. Em 220 V, valores da ordem de 2500 VA são usuais.
  \item \textit{TUG:} Limitar a potência total para manter correntes compatíveis com disjuntores de uso geral (ex.: 16 A ou 20 A). Em 220 V, limites da ordem de 4300 VA são referências comuns de projeto.
\end{itemize}

\subsection{Circuitos de reserva}
\label{subsec:ft_circuitos_reserva}

A ABNT NBR 5410 estabelece que os quadros de distribuição devem prever capacidade de reserva (espaço físico) para ampliações futuras \cite{NBR5410:2004}. Na prática de projeto, isso se materializa prevendo espaços ou circuitos reserva no quadro, dimensionados e deixados vagos para uso futuro. O número mínimo de circuitos de reserva é definido em função da quantidade de circuitos efetivos da instalação, conforme apresentado na Tabela \ref{tab:circuitos_reserva}.

\begin{table}[h!]
	\centering
	\Caption{\label{tab:circuitos_reserva} Espaço mínimo destinado à reserva em quadros de distribuição}
	\UFCtab{}{
		\begin{tabular}{cc}
			\toprule
			Quantidade de circuitos efetivos ($N$) & Espaço mínimo para reserva (nº de circuitos) \\
			\midrule \midrule
			$N \le 6$ & 2 \\
			$7 \le N \le 12$ & 3 \\
			$13 \le N \le 30$ & 4 \\
			$N > 30$ & $0{,}15 \cdot N$ \\
			\bottomrule
		\end{tabular}
	}{
		\Fonte{Adaptado de ABNT \cite{NBR5410:2004}.}
	}
\end{table}

\textbf{Nota:} Conforme prescrição normativa, a capacidade de reserva deve ser considerada no cálculo do alimentador do respectivo quadro de distribuição, assegurando que o sistema de entrada comporte ampliações previstas.

\section{Dimensionamento de condutores e proteções (NBR 5410)}
\label{sec:ft_dimensionamento_condutores}

Uma vez definidos os circuitos, procede-se ao dimensionamento dos condutores e à seleção dos dispositivos de proteção.

\subsection{Corrente de projeto, ampacidade e seção mínima}
\label{subsec:ft_corrente_ampacidade}

\paragraph{Corrente de projeto ($I_B$):} Calculada a partir da potência do circuito e da tensão ($I_B = S/V$).

\paragraph{Ampacidade ($I_Z$):}
A capacidade de condução de corrente do condutor é obtida inicialmente a partir das tabelas de ampacidade da NBR 5410, que fornecem um valor tabelado de referência (chamado aqui de $I_{Z,tab}$) para um dado método de instalação, material e tipo de isolação, considerando também o número de condutores carregados no circuito. Entretanto, as tabelas são definidas para condições padronizadas; quando as condições reais diferem (por exemplo, temperatura ambiente maior ou agrupamento de circuitos), a ampacidade deve ser corrigida por fatores multiplicativos. Assim, adota-se:

\begin{equation}
    I_Z = I_{Z,tab} \cdot k_T \cdot k_G \cdot k_I \cdot \dots
\end{equation}

onde:
\begin{itemize}
    \item $k_T$ é o fator de correção por temperatura ambiente (reduz $I_Z$ quando a temperatura é superior à de referência da tabela);
    \item $k_G$ é o fator de correção por agrupamento (reduz $I_Z$ quando há múltiplos circuitos/condutores próximos no mesmo eletroduto/canaleta, devido ao pior escoamento térmico);
    \item $k_I$ é o fator de correção por condição de instalação com pior dissipação térmica, quando aplicável (ex.: trechos em isolamento térmico, situações específicas definidas pela norma);
    \item ``$\dots$'' representa outros fatores normativos que só são aplicados quando o cenário exigir.
\end{itemize}

Na prática, o procedimento adotado é: (1) definir o método de instalação e as características do cabo (ex.: cobre/PVC e método B1), (2) obter $I_{Z,tab}$ na tabela correspondente, (3) aplicar os fatores pertinentes ao caso para obter $I_Z$, e então (4) selecionar a menor seção que satisfaça $I_Z \ge I_B$, mantendo a coordenação $I_B \le I_N \le I_Z$.

\paragraph{Seção mínima:} A norma impõe seções mínimas por razões mecânicas, independentemente da corrente calculada:
\begin{itemize}
  \item $1{,}5~\mathrm{mm}^2$ para iluminação;
  \item $2{,}5~\mathrm{mm}^2$ para circuitos de força (tomadas).
\end{itemize}

\subsection{Critério de coordenação: $I_B \le I_N \le I_Z$}
\label{subsec:ft_criterio_disjuntor}

Para proteção contra sobrecarga, deve-se atender à relação fundamental de coordenação:
$$ I_B \le I_N \le I_Z $$
onde $I_B$ é a corrente de projeto, $I_N$ é a corrente nominal do disjuntor e $I_Z$ é a capacidade de condução do condutor nas condições instaladas. Isso garante que o disjuntor não dispare em operação normal ($I_B \le I_N$) e que proteja o cabo antes que este sobreaqueça ($I_N \le I_Z$) \cite{Mamede2017}.

\subsection{Condutores de Proteção (PE) e Neutro}
\label{subsec:ft_pe_neutro}

\paragraph{PE (Terra):} Prover caminho de baixa impedância para faltas. Sua seção é definida normativamente com base na seção do condutor de fase.\paragraph{Neutro:} Condutor de retorno. Em circuitos monofásicos, sua seção é, em regra, igual à do condutor de fase.

\subsection{Hipóteses adotadas no escopo (220 V, B1, PVC)}
\label{subsec:ft_hipoteses}

Para viabilizar a automatização determinística neste trabalho, são adotadas hipóteses de projeto usuais para residências de padrão popular/médio na região do estudo:
\begin{itemize}
    \item Tensão de fase-neutro/fase-fase conforme concessionária (ex.: 220 V);
    \item Método de instalação B1 (condutores em eletroduto embutido em alvenaria);
    \item Condutores de Cobre com isolação PVC ($70^{\circ}$C);
    \item Consideração típica de 3 condutores carregados em agrupamentos iniciais (F+N ou F+F, agrupados), ajustável conforme o caso.
\end{itemize}

\section{Tipo de fornecimento e diretrizes da concessionária (ET-124 / ENEL Ceará)}
\label{sec:ft_tipo_fornecimento}

Além da NBR 5410, o projeto deve respeitar as normas da concessionária local (ET-124 da ENEL Ceará) \cite{ET124:ENEL}.

\subsection{Enquadramento do fornecimento pela potência instalada}
\label{subsec:ft_enquadramento}

A ET-124 define se o fornecimento será monofásico, bifásico ou trifásico com base na potência instalada total da unidade consumidora.
A potência instalada é a soma das potências atribuídas a todas as cargas (iluminação, TUG e TUE). Faixas de potência (ex.: até 15 kW, até 25 kW) determinam o tipo de atendimento.

\subsection{Impactos do tipo de fornecimento no projeto}
\label{subsec:ft_impacto_fornecimento}

O tipo de fornecimento afeta:
\begin{enumerate}
    \item A arquitetura do quadro (monofásico em comparação ao trifásico);
    \item O padrão de entrada e medição;
    \item A necessidade de balanceamento de fases na distribuição dos circuitos, visando equilibrar as cargas entre as fases disponíveis.
\end{enumerate}

\section{Inteligência Artificial e Modelos de Linguagem (LLM)}
\label{sec:ft_llm}

A Inteligência Artificial (IA) passou por uma transformação paradigmática nos últimos anos com o advento dos Modelos de Linguagem de Grande Escala (LLMs, ou \textit{Large Language Models}). Estes modelos representam o estado da arte em Processamento de Linguagem Natural (PLN), demonstrando capacidades surpreendentes de generalização, raciocínio lógico e geração de código.

\subsection{Arquitetura Transformer e Mecanismo de Atenção}
\label{subsec:ft_transformer}

A base tecnológica dos LLMs modernos é a arquitetura \textit{Transformer}, introduzida por \citeonline{Vaswani2017} no trabalho seminal ``\textit{Attention Is All You Need}''. Diferentemente das arquiteturas anteriores baseadas em Redes Neurais Recorrentes (RNNs) e LSTMs (\textit{Long Short-Term Memory}), que processavam dados sequencialmente, o Transformer permite o processamento paralelo de toda a sequência de entrada.

O componente central dessa arquitetura é o mecanismo de \textit{autoatenção} (\textit{self-attention}). Esse mecanismo permite que o modelo pondere a relevância de cada palavra (ou \textit{token}) em relação a todas as outras na mesma sequência, independentemente da distância entre elas. Matematicamente, a atenção pode ser descrita como uma função que mapeia uma consulta ($Q$), uma chave ($K$) e um valor ($V$) para uma saída:

\begin{equation}
    \text{Attention}(Q, K, V) = \text{softmax}\left(\frac{QK^T}{\sqrt{d_k}}\right)V
\end{equation}

Onde $d_k$ é a dimensão da chave. Essa capacidade de capturar dependências de longo prazo e contextos complexos é o que permite aos LLMs compreenderem nuances da linguagem humana, referências cruzadas e instruções compostas.

\subsection{Pré-treinamento e Ajuste Fino (Fine-Tuning)}

O ciclo de vida de um LLM envolve tipicamente duas etapas principais:

\begin{enumerate}
    \item \textbf{Pré-treinamento:} O modelo é treinado em um \textit{corpus} massivo de texto (trilhões de palavras provenientes da web, livros, artigos e código-fonte) com o objetivo de prever o próximo \textit{token} em uma sequência. Nesta fase, o modelo adquire conhecimento geral sobre o mundo, gramática, lógica e até mesmo fundamentos de programação.
    \item \textbf{Ajuste Fino (\textit{Fine-Tuning}):} O modelo pré-treinado é refinado em conjuntos de dados menores e mais específicos, muitas vezes utilizando técnicas como RLHF (\textit{Reinforcement Learning from Human Feedback}) para alinhar o comportamento do modelo às instruções humanas, tornando-o mais útil, seguro e assertivo para tarefas de assistência \cite{Nguyen2025}.
\end{enumerate}

\subsection{Limitações: Alucinações e Janela de Contexto}

Apesar de seu poder, os LLMs possuem limitações críticas para aplicações de engenharia:

\begin{itemize}
    \item \textbf{Alucinações:} Como modelos probabilísticos, os LLMs podem gerar informações factualmente incorretas, mas com alta confiança e fluidez \cite{Ji2023}. Em um projeto elétrico, inventar uma norma ou um valor de corrente é inaceitável.
    \item \textbf{Janela de Contexto Limitada:} Embora tenha crescido, a quantidade de informação que um modelo pode processar de uma vez (sua "memória de curto prazo") é finita. Não é possível, por exemplo, inserir o texto integral de todas as normas técnicas brasileiras no \textit{prompt} de uma única vez.
    \item \textbf{Dificuldade com Aritmética:} LLMs são excelentes em lógica linguística, mas frequentemente falham em cálculos matemáticos precisos, a exemplo de somas complexas ou operações com ponto flutuante, pois tentam prever o resultado token por token em vez de calcular.
\end{itemize}

\textbf{Aplicação neste trabalho:} Reconhecendo essas limitações, a metodologia proposta utiliza o LLM estritamente como um orquestrador semântico. O modelo é responsável por interpretar os requisitos do usuário (como a solicitação de um chuveiro de 5500W) e estruturar o raciocínio, mas os cálculos de engenharia e as verificações normativas são delegados a ferramentas externas determinísticas, garantindo que a "alucinação" não afete a segurança do projeto elétrico.

\section{Recuperação Aumentada por Busca (RAG)}
\label{sec:ft_rag}

Para mitigar o problema das alucinações e a limitação de conhecimento atualizado, a técnica de \textit{Retrieval-Augmented Generation} (RAG) tem se consolidado como padrão na indústria \cite{Lewis2020}.

\subsection{Conceito e Comparação com Fine-Tuning}

O RAG é uma arquitetura que combina um modelo gerador (o LLM) com um módulo recuperador (\textit{Retriever}), conforme ilustrado na Figura \ref{fig:rag}. Em vez de confiar apenas nos parâmetros internos do modelo (sua "memória paramétrica"), o sistema busca informações relevantes em uma base de conhecimento externa confiável antes de gerar a resposta.

\begin{figure}[h!]
	\centering
	\Caption{\label{fig:rag} Arquitetura básica de um sistema RAG}
	\UFCfig{}{
		\includegraphics[width=0.8\textwidth]{figuras/rag.png}
	}{
		\Fonte{Elaborado pelo autor.}
	}
\end{figure}

É fundamental distinguir o RAG do \textit{Fine-Tuning}. Enquanto o \textit{Fine-Tuning} é utilizado para adaptar o estilo, o comportamento ou a forma como o modelo responde (sendo ideal para tarefas repetitivas ou de domínio muito específico), o RAG é preferível quando o conhecimento necessário é vasto, frequentemente atualizado e requer citação explícita da fonte. No contexto de normas técnicas, onde a auditabilidade e a precisão do texto original são cruciais, o RAG é a abordagem mais adequada.

\subsection{Implementação Prática: Indexação, Recuperação e Geração}

A aplicação de um sistema RAG eficaz envolve etapas técnicas bem definidas:

\begin{itemize}
    \item \textbf{Indexação e \textit{Chunking}:} Documentos longos, como normas em PDF, são pré-processados e divididos em fragmentos menores (\textit{chunks}), respeitando a estrutura lógica do documento (por exemplo, por seções ou artigos da norma). Metadados como número do item e página são preservados para rastreabilidade.
    \item \textbf{Embeddings e Busca Vetorial:} Cada fragmento de texto é convertido em um vetor numérico (\textit{embedding}) de alta dimensão que representa seu significado semântico. A recuperação utiliza algoritmos de busca de similaridade (como similaridade de cosseno) para encontrar os $k$ trechos (\textit{top-k}) mais relevantes para a consulta do usuário, mesmo que as palavras exatas não coincidam.
    \item \textbf{Montagem de Contexto (\textit{Context Assembly}):} Os trechos recuperados são injetados no \textit{prompt} do sistema. O modelo recebe uma instrução explícita para usar apenas o contexto fornecido abaixo para responder.
    \item \textbf{\textit{Grounding} e Rastreabilidade:} Para garantir a veracidade, o modelo é instruído a citar a fonte da informação utilizada (por exemplo, ``conforme item 6.2 da NBR 5410 recuperado''), permitindo que o engenheiro humano verifique a base da decisão.
\end{itemize}

\textbf{Aplicação neste trabalho:} A base de conhecimento do sistema proposto é composta pelos textos da NBR 5410 e da ET-124. Através do RAG, o agente recupera os critérios exatos (como fatores de agrupamento e seções mínimas) no momento da decisão, assegurando que o memorial de cálculo esteja ancorado na norma vigente e não em conhecimento genérico ou alucinado.

\section{Agentes Inteligentes e Sistemas Autônomos}
\label{sec:ft_agentes}

Enquanto um sistema RAG é passivo, pois responde a uma pergunta, um \textbf{Agente} é um sistema autônomo capaz de perseguir objetivos complexos. \citeonline{Yao2023} definem agentes baseados em LLM como sistemas onde o modelo de linguagem atua como o "cérebro", responsável pelo planejamento e raciocínio, enquanto interfaces externas fornecem a capacidade de ação.

\subsection{O Padrão ReAct (Reasoning + Acting)}

Uma das abordagens mais eficazes para a construção de agentes é o padrão \textit{ReAct} (Synergizing Reasoning and Acting in Language Models). Onde modelos tradicionais apenas geram texto, um agente ReAct opera em um loop contínuo, ilustrado na Figura \ref{fig:react}, composto por:

\begin{figure}[h!]
	\centering
	\Caption{\label{fig:react} Ciclo de raciocínio e ação no padrão ReAct}
	\UFCfig{}{
		\includegraphics[width=0.6\textwidth]{figuras/react.jpg}
	}{
		\Fonte{Adaptado de \citeonline{Yao2023}.}
	}
\end{figure}

\begin{itemize}
    \item \textbf{Pensamento (\textit{Thought}):} O agente analisa o estado atual e decide o que precisa ser feito (como a necessidade de calcular a corrente do circuito do chuveiro).
    \item \textbf{Ação (\textit{Action}):} O agente escolhe uma ferramenta para executar uma tarefa específica (por exemplo, chamar a função \texttt{calcular\_corrente(potencia=5500, tensao=220)}).
    \item \textbf{Observação (\textit{Observation}):} O agente recebe o resultado da ferramenta e o incorpora ao seu raciocínio para dar o próximo passo.
\end{itemize}

\subsection{Confiabilidade e Controle: Ferramentas Determinísticas}

Para aplicações de engenharia, a autonomia do agente deve ser balizada por rigor técnico. Isso é alcançado através do uso de ferramentas (Figura \ref{fig:tools}):

\begin{figure}[h!]
	\centering
	\Caption{\label{fig:tools} Interação entre o LLM e ferramentas externas}
	\UFCfig{}{
		\includegraphics[width=0.6\textwidth]{figuras/tools.jpg}
	}{
		\Fonte{Elaborado pelo autor.}
	}
\end{figure}

\begin{itemize}
    \item \textbf{Ferramentas Determinísticas:} As funções que o agente chama (cálculo de queda de tensão, seleção de disjuntor) são algoritmos de software clássico, testáveis e imunes a alucinações.
    \item \textbf{Critérios de Parada e Validação:} O agente opera sob restrições lógicas. Por exemplo, uma ferramenta de seleção de cabos verifica internamente a condição $I_B \le I_N \le I_Z$ e retorna erro caso não satisfeita, forçando o agente a revisar seu plano antes de prosseguir.
\end{itemize}

\textbf{Aplicação neste trabalho:} O agente desenvolvido executa um loop ReAct onde cada etapa do projeto elétrico (levantamento de carga, divisão de circuitos, dimensionamento) é uma ação que invoca uma função de engenharia específica. O memorial de cálculo final não é uma criação livre do LLM, mas sim o registro estruturado das saídas validadas dessas ferramentas determinísticas, orquestradas pela inteligência do modelo.

\section{Síntese do capítulo}
\label{sec:ft_sintese}

Este capítulo fundamentou os dois pilares do trabalho. De um lado, a engenharia elétrica, com as prescrições rígidas da ABNT NBR 5410 e da ET-124, que definem \textit{o que} deve ser feito. Do outro, a Inteligência Artificial moderna, através de LLMs, RAG e Agentes ReAct, que definem \textit{como} automatizar esse processo de forma inteligente. A estratégia de separar o determinismo (cálculo) da fluidez (texto) é fundamental para garantir que o memorial de cálculo gerado seja, ao mesmo tempo, tecnicamente rigoroso e acessível ao usuário.
